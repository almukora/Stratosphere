\documentclass[10pt]{article}
\usepackage[table]{xcolor}
\usepackage{fancyhdr}
\usepackage{amsmath} 
\usepackage[margin=3cm]{geometry}
\usepackage{graphicx} 
\usepackage{longtable}  
\usepackage{multicol} 
\usepackage{float}
\usepackage{xparse}
\usepackage{amssymb}
\newcommand{\degrees}{\ensuremath{^\circ}}
\usepackage{wrapfig}
\usepackage{parskip}
\usepackage{listings}
\usepackage{enumerate}
\newcommand{\csection}[1]{\section[#1]{\centering #1}}
\newcommand{\csubsection}[1]{\subsection[#1]{\centering #1}}


\newcommand{\code}[1]{\lstinline{#1}}
\renewcommand{\d}{\ensuremath{\operatorname{d}\!}}
\newcommand{\cin}[1]{{\scriptsize \lstinputlisting{#1}}}

\newcommand{\del}{\vec\nabla}
\newcommand{\p}{\partial}
\newcommand{\E}[1]{\times10^{#1}}
\newcommand{\eq}[1]{\begin{equation*}#1\end{equation*}}
\newcommand{\al}[1]{\begin{align*}#1\end{align*}}
\newcommand{\aln}[1]{\begin{align}#1\end{align}}
\newcommand{\sub}[1]{_\text{#1}}
% \newcommand{\qed}{\hfill \ensuremath{\Box}}
\newcommand{\vv}{\mathbf}
% \newcommand{\qed}{\hfill \ensuremath{\Box}~~~~~~~}

\newcommand{\dd}[2]{\frac{\d\,#1}{\d\,#2}}
\newcommand{\ddo}[1]{\frac{d}{d\,#1}}
\newcommand{\ddt}[1]{\frac{\d\,#1}{\d\,t}}
\newcommand{\dds}[2]{\frac{\d^2\,#1}{\d\,#2^2}}
\newcommand{\pd}[2]{\frac{\partial \,#1}{\partial \,#2}}
\newcommand{\spd}[2]{\frac{\partial^2 #1}{\partial #2^2}}
\newcommand{\lp}{\left(}
\newcommand{\rp}{\right)}

\newcommand{\lb}{\left[}
\newcommand{\rb}{\right]}


\usepackage{color}

\definecolor{dkgreen}{rgb}{0,0.6,0}
\definecolor{gray}{rgb}{0.5,0.5,0.5}
\definecolor{mauve}{rgb}{0.58,0,0.82}

\lstset{
  language=C,          
  % backgroundcolor=\color{gray}
  keywordstyle=\color{blue}\sffamily\bfseries,  
  commentstyle=\color{dkgreen}\itshape,    
  backgroundcolor=\color{Yellow!25},
  basicstyle=\sffamily,
  % basicstyle=\footnotesize,  
  numbers=left,              
  numberstyle=\footnotesize,      
  stepnumber=5,                   
  numbersep=5pt,                  
  backgroundcolor=\color{white},  
  showspaces=false,               
  showstringspaces=false,         
  showtabs=false,                 
  frame=single,           
  framerule=0.2pt,%expands outward 
  % rulecolor=\color{red},
  % framesep=3pt,%expands outward
  tabsize=4,          
  captionpos=b,           
  breaklines=true,        
  breakatwhitespace=false,    
  xleftmargin=3.2pt,
  xrightmargin=15pt,
}

\definecolor{lightblue}{rgb}{0.83,0.85,1.0}

\newcommand{\shadetable}[5]
{
  \begin{table}[H] 
    \begin{center}
      \caption{#3} \vspace{5pt}
      \rowcolors{1}{}{lightblue}
      \begin{tabular}{#1} \hline 
        #2 \\
        \hline
        #4 \\
        \hline                       
      \end{tabular}
      \label{#5}
    \end{center}
  \end{table}
}
% \shadetable{c|c}{column1 & column2}
% {Test Table 1}{  
%   a & b \\
%   c & d
% }{tab:test1}



\DeclareDocumentCommand\m{ m g g g g g g g g}
  {
    \left( 
      \begin{smallmatrix} 
        #1
        \IfNoValueF{#2}{\\#2}
        \IfNoValueF{#3}{\\#3}
        \IfNoValueF{#4}{\\#4}
        \IfNoValueF{#5}{\\#5}
        \IfNoValueF{#6}{\\#6}
        \IfNoValueF{#7}{\\#7}
        \IfNoValueF{#8}{\\#8}
        \IfNoValueF{#9}{\\#9}
      \end{smallmatrix} 
    \right)
}
% \m{row1}[{row2}][{...}{...}...]

\newcommand{\fig}[4]
{
  \begin{figure}[h] \centering \vspace{-10pt} 
    \includegraphics[width=#4\textwidth]{#1} \vspace{-10pt} 
    \caption{#2} \label{#3}
  \end{figure}
}
% \figH {FILENAME}{CAPTION}{LABEL}{WIDTH}       

\newcommand{\figH}[4]
{
  \begin{figure}[H] \centering \vspace{-10pt} 
    \includegraphics[width=#4\textwidth]{#1}\vspace{-10pt} 
    \caption{#2} \label{#3}
  \end{figure}
} 
% \figH {FILENAME}{CAPTION}{LABEL}{WIDTH}       


% \AtBeginDocument{%
  % \setlength\abovedisplayskip{0pt}
  % \setlength\belowdisplayskip{0pt}}

% ------------------------------------------------------------------------
%                               Page Style
% ------------------------------------------------------------------------

\pagestyle{fancy}
\renewcommand{\sectionmark}[1]{\markright{\thesection\ #1}}
\fancyhf{}

\lhead{\fancyplain{}{\rightmark }}
\cfoot{\fancyplain{}{\thepage}}


\usepackage{titlesec}
% \titleformat{\section}[block]{\centering\Large\bfseries\filcenter}{}{1em}{}
\titleformat{\subsection}[hang]{\centering\bfseries}{}{1em}{}

\title{
  Stratosphere \\ 
  \small{A multilevel data-based forum for cloud computing}
} 

\author{
  Miller, Joshua \\
  Claxton, Spencer \\
  Mukora, Alice \\
  Fleming, Zac
}

\date{\small{18 June 2013}}


\begin{document}
\rhead{\fancyplain{}{Stratosphere Proposal}}

\maketitle

\begin{abstract}

  Stratosphere is a data-based collection of tools for connecting
  cloud researchers amongst disciplines.  The method for doing this
  includes data access records, peer authentication of user
  generated data, and making users and their research visible to other users.

  We are attempting to sell an idea here. We have no proof of concept
  save the precedent of other social networks and the benefits they offer.

\end{abstract}

\tableofcontents
\newpage
\begin{multicols*}{2}
\section{What is Stratosphere}

Stratosphere is a data-based social-network-like system with the goal
of connecting researchers in different groups and disciplines.
Stratosphere is composed of two layers, the user interaction layer and
the data interaction layer.  The user interaction layer is a
superstructure designed with social networking in mind to create a
social interdisciplinary network centered around the data.  The data
interaction layer handles user access to data. Stratosphere provides a
peer approval system for file permissions of user submitted data and
informs the users of who is using the data and with what intent.

\subsection{Why Stratosphere}

Stratosphere attempts to address the problem of inter-connectivity
between scientists using the OSDC system.  Currently, researchers are
only connected to each other a priori, by institution or by prior
collaboration.  The core concept of scientific cloud computing in an
open setting is the promotion of collaboration and multifaceted
approaches to data analysis.  Users are currently cut off from peer
collaboration on research projects conducted via the cloud. By
distributing the responsibility of project documentation and making
the users visible to one another, the OSDC would promote direct
interaction with researchers from different projects who desire to
work on the same data set.

\subsection{A Data-based network}

The main infrastructure of Stratosphere is data based.  This means
that user access is documented in a metadata descriptor attached to
each data set. For example, user access to the dataset might tally a
rate counter specific to that individual user.  Using this rate, we
would be able to present a list of top users for each dataset to other
users interested in the data. Furthermore, data frequently used by an
individual would be represented as specializations on the Stratosphere
front end. The metadata includes username, last access, number of
accesses, project, project description, and contact.


\figH {metaRecording.png}{Process of recording user access metadata}{metarecord}{.5}       

\subsection{Peer-Authentication}

This social-interactive system aids the moderation of user
permissions.  When users add data to the cloud they are able to assign
ownership to the data set, with the default being public
ownership. Publicly owned data sets require no permissions to access.
Should the creator of a data set request ownership, then he,she, or
selected other users take the responsibility of being an
administrative user.  Users are then granted access to the data set by
the administrative user.

\subsection{Visibility and Privacy}

The main goal of Stratosphere is to provide transparency through the
cloud. To do this we propose that users be given public profiles that
display the data they are working with and their contact information.
However, we are well aware that some actions need not be recorded in a
user's data access. For this reason, we propose the use of defaults
which lean towards full record keeping and visibility, with the option
to remove such visibility.

\section{Implementation Challenges}

\subsection{User-to-Data Interactions}

The user to data interaction is the most challenging aspect of this
project.  The tools that need to be implemented for this include the
implementation of
\begin{itemize}
\item file monitoring for user access records
\item peer authenticated permissions
\end{itemize}

\textbf{File monitoring}

Possibilities for monitoring the frequency of data access include
tools such as \textit{inotify}, or direct patching of the file
mounting, or other proprietary/open software.  Preliminary testing of
accepted tools such as inotify seem improbable to use.  Starting the
service ran at approximately 16 GB/s for an 80GB section of the open
data. This appears to be non-scalable (~70 hours to index the entire
cloud). Therefore, we propose a patch to the mounting system such that
each new instance of a file access increments the metadata use
counter.  The main assumption is that the users who access the data
the most have the highest interest and experience with the particular
data set.

\textbf{Metadata creation}

We propose to include metadata based on who accesses the data, how
often they access the data, and for what purpose they are accessing
the data.  In order to maintain privacy, the user is not required, but
encouraged to assign a purpose comment or project to each data set
accessed. This data set descriptor persists for each user until
changed.  When the data is accessed, it is the job of the file
mounting system to intercept and record the request.  We have provided
a proof of concept implementation (using XML metadata) of a file
system mount adapted from \textbf{fusedev} for this purpose.

\textbf{Permissions}

The task of creating peer authenticated permissions is a far more
challenging problem. Currently, the OSDC cloud operates on basic Unix
group permissions, and due to the binary grouping of open data and
protected data, this works for now. However, we would be forced to
store the entire combination of each user inclusion in each research
permission group. A file system with \textit{Dropbox}-style
permissions is certainly preferable.However, we currently have no
recommendation for how such a file system would be implemented so that
it is big-data scalable.

\subsection{User-to-User Interaction}

The main aspect of interactions between users in the Stratosphere is
the ability of the metadata to describe itself in terms of
users. Users are able to see what peers are doing with the data of
interest and are further able to get to a full research profile of the
users that are working with the same data. To make this possible we
propose the implementation of user profiles that are accessible from
and linked to the data that the user has accessed. The biggest
challenge here would be integrating such a social network with the
current OSDC interface (the Tukey portal).


\end{multicols*}
\end{document}
