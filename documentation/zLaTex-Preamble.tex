\documentclass[10pt]{article}
\usepackage[table]{xcolor}
\usepackage{fancyhdr}
\usepackage{amsmath} 
\usepackage[margin=3cm]{geometry}
\usepackage{graphicx} 
\usepackage{longtable}  
\usepackage{multicol} 
\usepackage{float}
\usepackage{xparse}
\usepackage{amssymb}
\newcommand{\degrees}{\ensuremath{^\circ}}
\usepackage{wrapfig}
\usepackage{parskip}
\usepackage{listings}
\usepackage{enumerate}
\newcommand{\csection}[1]{\section[#1]{\centering #1}}
\newcommand{\csubsection}[1]{\subsection[#1]{\centering #1}}


\newcommand{\code}[1]{\lstinline{#1}}
\renewcommand{\d}{\ensuremath{\operatorname{d}\!}}
\newcommand{\cin}[1]{{\scriptsize \lstinputlisting{#1}}}

\newcommand{\del}{\vec\nabla}
\newcommand{\p}{\partial}
\newcommand{\E}[1]{\times10^{#1}}
\newcommand{\eq}[1]{\begin{equation*}#1\end{equation*}}
\newcommand{\al}[1]{\begin{align*}#1\end{align*}}
\newcommand{\aln}[1]{\begin{align}#1\end{align}}
\newcommand{\sub}[1]{_\text{#1}}
% \newcommand{\qed}{\hfill \ensuremath{\Box}}
\newcommand{\vv}{\mathbf}
% \newcommand{\qed}{\hfill \ensuremath{\Box}~~~~~~~}

\newcommand{\dd}[2]{\frac{\d\,#1}{\d\,#2}}
\newcommand{\ddo}[1]{\frac{d}{d\,#1}}
\newcommand{\ddt}[1]{\frac{\d\,#1}{\d\,t}}
\newcommand{\dds}[2]{\frac{\d^2\,#1}{\d\,#2^2}}
\newcommand{\pd}[2]{\frac{\partial \,#1}{\partial \,#2}}
\newcommand{\spd}[2]{\frac{\partial^2 #1}{\partial #2^2}}
\newcommand{\lp}{\left(}
\newcommand{\rp}{\right)}

\newcommand{\lb}{\left[}
\newcommand{\rb}{\right]}


\usepackage{color}

\definecolor{dkgreen}{rgb}{0,0.6,0}
\definecolor{gray}{rgb}{0.5,0.5,0.5}
\definecolor{mauve}{rgb}{0.58,0,0.82}

\lstset{
  language=C,          
  % backgroundcolor=\color{gray}
  keywordstyle=\color{blue}\sffamily\bfseries,  
  commentstyle=\color{dkgreen}\itshape,    
  backgroundcolor=\color{Yellow!25},
  basicstyle=\sffamily,
  % basicstyle=\footnotesize,  
  numbers=left,              
  numberstyle=\footnotesize,      
  stepnumber=5,                   
  numbersep=5pt,                  
  backgroundcolor=\color{white},  
  showspaces=false,               
  showstringspaces=false,         
  showtabs=false,                 
  frame=single,           
  framerule=0.2pt,%expands outward 
  % rulecolor=\color{red},
  % framesep=3pt,%expands outward
  tabsize=4,          
  captionpos=b,           
  breaklines=true,        
  breakatwhitespace=false,    
  xleftmargin=3.2pt,
  xrightmargin=15pt,
}

\definecolor{lightblue}{rgb}{0.83,0.85,1.0}

\newcommand{\shadetable}[5]
{
  \begin{table}[H] 
    \begin{center}
      \caption{#3} \vspace{5pt}
      \rowcolors{1}{}{lightblue}
      \begin{tabular}{#1} \hline 
        #2 \\
        \hline
        #4 \\
        \hline                       
      \end{tabular}
      \label{#5}
    \end{center}
  \end{table}
}
% \shadetable{c|c}{column1 & column2}
% {Test Table 1}{  
%   a & b \\
%   c & d
% }{tab:test1}



\DeclareDocumentCommand\m{ m g g g g g g g g}
  {
    \left( 
      \begin{smallmatrix} 
        #1
        \IfNoValueF{#2}{\\#2}
        \IfNoValueF{#3}{\\#3}
        \IfNoValueF{#4}{\\#4}
        \IfNoValueF{#5}{\\#5}
        \IfNoValueF{#6}{\\#6}
        \IfNoValueF{#7}{\\#7}
        \IfNoValueF{#8}{\\#8}
        \IfNoValueF{#9}{\\#9}
      \end{smallmatrix} 
    \right)
}
% \m{row1}[{row2}][{...}{...}...]

\newcommand{\fig}[4]
{
  \begin{figure}[h] \centering \vspace{-10pt} 
    \includegraphics[width=#4\textwidth]{#1} \vspace{-10pt} 
    \caption{#2} \label{#3}
  \end{figure}
}
% \figH {FILENAME}{CAPTION}{LABEL}{WIDTH}       

\newcommand{\figH}[4]
{
  \begin{figure}[H] \centering \vspace{-10pt} 
    \includegraphics[width=#4\textwidth]{#1}\vspace{-10pt} 
    \caption{#2} \label{#3}
  \end{figure}
} 
% \figH {FILENAME}{CAPTION}{LABEL}{WIDTH}       


% \AtBeginDocument{%
  % \setlength\abovedisplayskip{0pt}
  % \setlength\belowdisplayskip{0pt}}

% ------------------------------------------------------------------------
%                               Page Style
% ------------------------------------------------------------------------

\pagestyle{fancy}
\renewcommand{\sectionmark}[1]{\markright{\thesection\ #1}}
\fancyhf{}

\lhead{\fancyplain{}{\rightmark }}
\cfoot{\fancyplain{}{\thepage}}
